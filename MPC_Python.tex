%%% Beamber preamble - June 2015 %%%
% FR version

\documentclass[t]{beamer} % Top-aligned slides

%% Font & language
\usepackage[utf8x]{inputenc}
\usepackage[frenchb]{babel}
\usepackage[T1]{fontenc}
\usepackage{lmodern}
%\PrerenderUnicode{éàèê}

%% Packages: math, symbols
%\usepackage{gensymb} % for \celsius command
\usepackage{amsmath}
\usepackage{amssymb}
\usepackage{textcomp} % Symbole Euro €
\newcommand{\euro}{\texteuro}
%\usepackage{fixltx2e} % for \textsubscript command


%% Beamer customization:
\usetheme{Frankfurt}
\usecolortheme{beaver} % red & gray colors
% enlever l'ombre sous le titre: (http://tex.stackexchange.com/a/3182/51609)
\setbeamertemplate{title page}[default][colsep=-4bp,rounded=true]
% enlever l'ombre sous toutes les boîtes
\setbeamertemplate{blocks}[rounded][shadow=false]

%\setbeamertemplate{mini frames}{} % remove subsection circles

% Puces :
\setbeamertemplate{itemize items}{$\circ$}
\setbeamertemplate{enumerate items}[default]
\setbeamertemplate{section in toc}[sections numbered]
\setbeamertemplate{subsection in toc}[square]

% Frame number
\setbeamertemplate{footline}[frame number]
\setbeamertemplate{sidebar right}{} % (removes the navigation bar)




%% Custom commands

% Mathematical Expectation E[X] and Covariance
\newcommand\E[1]{\mathbb{E}[#1]}
\newcommand\avg[1]{\langle #1 \rangle} % ⟨X⟩ notation
\DeclareMathOperator{\cov}{Cov}
\DeclareMathOperator{\var}{Var}

% Absolute value and norm (double bar)
\providecommand{\abs}[1]{\lvert#1\rvert}
\providecommand{\norm}[1]{\lVert#1\rVert}

\newcommand\detail[1]{{\color{gray} \footnotesize #1}}


%% Colors
\usepackage{xcolor}
% general purpose:
\definecolor{green}{rgb}{0,0.5,0}
\definecolor{bourgogne}{rgb}{0.76,0,0.40}
\definecolor{turquoise}{rgb}{0,0.55,0.56}
\definecolor{darkblue}{rgb}{0.2,0.4,0.7}
% SDP colors
\definecolor{present}{rgb}{1,0.4,0} % instant cost highlight
\definecolor{future}{rgb}{0.2,0.2,0.7} % future cost highlight
\definecolor{underbrace}{gray}{0.65}


% Affiche en début de chaque section pas trop gros (\small),
% les noms des sections, celle en cours en évidence,
% les autres en grisé et les noms des sous-sections de la section en cours uniquement.
\AtBeginSection[]{
  \begin{frame}{Plan de la présentation}
  \small \tableofcontents[currentsection, hideothersubsections]
  \end{frame}
}


\title{Commande prédictive avec Python.\\
Application au pilotage optimal du chauffage d’un bâtiment.}

\author{Pierre Haessig, Sylvain Chatel, Romain Bourdais}
% add Amanda, Hervé
\institute{
    CentraleSupélec -- IETR
    }
\date{PyCon, Rennes, 16 octobre 2016}


\begin{document}
%------- page de titre --------
  \begin{frame}

  \titlepage

  \color{gray} \small
   \url{http://pierreh.eu}
   \hfill
   \texttt{pierre.haessig@centralesupelec.fr}

  \end{frame}

% --------- Sommaire ---------
 \begin{frame}
   \frametitle{Plan de la présentation}

   \tableofcontents

 \end{frame}
% ----------------------------

\section{Enjeux}

\begin{frame}
  \frametitle{Énergie thermique dans le bâtiment}

  \begin{block}{Enjeu énergétique et environnemental}
    Chauffer/refroidir un bâtiment consomme \emph{beaucoup} d'énergie. %TODO : bilan énergétique france
  \end{block}



  \bigskip

  Pour réduire cette consommation, deux types de leviers:

  \begin{itemize}
    \item améliorer la \textbf{structure} (isolation, ...) du bâtiment\\
    → ``\emph{hardware} upgrade''
    \pause
    \item améliorer la \textbf{commande} (pilotage) du chauffage/clim.\\
    → ``\emph{software} upgrade''
    \uncover{\color{bourgogne}[sujet du jour]}
  \end{itemize}

\end{frame}


\section{Commande de chauffage classique}

\begin{frame}
  \frametitle{Choix du pas de temps}
  
  La performance des systèmes énergétiques dynamiques
  \detail{(véhicule en mouvement, système de stockage d'énergie)},
  est souvent évaluée par \textbf{simulation temporelle}.
  
  \bigskip
  
  → Le choix du pas de temps $\Delta_t$ relève d'un compromis :
  %
  \begin{itemize}
    \item $\Delta_t$ court : simulation fidèle, mais lente (temps de calcul)
    \item $\Delta_t$ long : simulation plus rapide, mais plus grossière
  \end{itemize}

\end{frame}

\section{Commande prédictive}
\subsection{les ingrédients}
\subsection{sur un Rasberry Pi}

\section{Conclusion}

\subsection{Choix du pas de temps}


\begin{frame}
  \frametitle{Conclusion : choix du pas de temps}
  \framesubtitle{sur notre exemple de système éolien-stockage}
  
  \begin{block}{}
    Le pas de temps de 10 min (couramment choisi),
    donne des résultats très proches de ceux obtenus avec un pas très fin.
  \end{block}

  \bigskip
  
  Résultat valable :
  \begin{itemize}
    \item pour la plupart des statistiques utiles au dimensionnement
    (écarts RMS et MA).
    \item y compris la durée de vie en cyclage (étonnamment ?)
    \item cependant le micro cyclage est totalement masqué.
  \end{itemize}

\end{frame}

\subsection{Généralisation et perspectives}


\begin{frame}
  \frametitle{Généralisation à d'autres systèmes}
  
  \begin{block}{}
    La question de la généralisation reste ouverte.
  \end{block}

  \bigskip
  → Applications où un stockage doit garantir une production EnR
  sur des horizons plus longs (typiquement J+1), l'effet du pas de temps devrait
  être encore plus faible.

  \bigskip
  → Par contre, un cas plus dépendant des fluctuations rapides
  comme le ``lissage de rampe'' pourrait être plus sensible.
  

\end{frame}

\begin{frame}
  \frametitle{Perspective}
  
  Dans des applications :
  
  \begin{itemize}
    \item qui dépendent des fluctuations haute fréquence,
    \item mais où l'on n'a que des données basse fréquence,
  \end{itemize}
  
  on pourrait essayer de \textbf{synthétiser un bruit haute fréquence} ?
  
  \begin{equation*}
    P_{HF}^{synth} = P_{BF}^{data} + b_{HF}^{synth}
  \end{equation*}
  
  \bigskip
  \emph{→ question du réalisme statistique du résultat obtenu...}

\end{frame}


\end{document}
